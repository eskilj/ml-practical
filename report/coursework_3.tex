\documentclass[]{article}

%opening
\title{Planning and Risks}
\author{Exam Number: B032374}
\date{\today}

\usepackage[parfill]{parskip}
\usepackage{pgfgantt}
\usepackage{multicol}
\usepackage{tikz}
\usepackage{pgfplots}
\begin{document}

\maketitle

\section{Introduction}

\section{Methodology}

\subsection{Activation Functions}

relu
elu
sigmoid
tanh

\begin{figure}[h]
	\centering
	\begin{tikzpicture}
	\begin{axis}[
	axis lines = left,
	xlabel = $x$,
	ylabel = {$f(x)$},
	xmin=-4, xmax=4,
	ymin=-2, ymax=3,
	]
	\addplot[color=red]{max(0, x)};
	\end{axis}
	\end{tikzpicture}
	\begin{tikzpicture}
	\begin{axis}[
	axis lines = left,
	xlabel = $x$,
	ylabel = {$f(x)$},
	xmin=-4, xmax=4,
	ymin=-2, ymax=3,
	]
	\addplot[color=blue]{tanh(x)};
	\end{axis}
	\end{tikzpicture}
	\begin{tikzpicture}
	\begin{axis}[
	axis lines = left,
	xlabel = $x$,
	ylabel = {$f(x)$},
	xmin=-4, xmax=4,
	ymin=-2, ymax=3,
	]
	\addplot[color=brown]{1/(1+e^(-x))};
	\end{axis}
	\end{tikzpicture}
	\caption{Showing the plots of the transformation functions. Left: Relu. Middle: Tanh. Right: Sigmoid.}
	\label{part1-plots}
\end{figure}

\subsection{Hidden Layers}

depths 1 2 4 5
widths 50 100 200 400



\subsection{Learning Rate Schedules}

\section{Results and Discussion}

\section{Conclusion}

\section{Future Work}


\end{document}
