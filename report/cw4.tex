\documentclass[]{article}

%opening
\title{MLP Coursework 4}
\author{Eskil Joergensen}
\date{\today}

\usepackage[parfill]{parskip}
\usepackage{pgfgantt}
\usepackage{multicol}
\usepackage{tikz}
\usepackage{pgfplots}

\usepackage{graphicx}
\graphicspath{ {images/} }

\usepackage{amsmath}

\usepackage{booktabs}
\usepackage[font={small,it}]{caption}

\begin{document}

\maketitle

\section{Introduction}

Optimizing the training performance of a convolutional neural network (CNN) on image classification task, CIFAR 10. Building on a baseline set of experiments, done with a deep neural network, the presented () tries to improve the performance of the more advanced methods used here. 

In addition to improving the overall performance of the training of the network, the experiments were guided by a goal of 70\% accuracy on the test data set. As a result, the methodology and incremental improvements to the model were made on these grounds. 

\section{Methodology}

Establishing a baseline model, and choosing the appropriate algorithms to include, can be a daunting task. Even a relatively small CNN has a large number of tunable parameters, and it can be difficult to know where to begin. The section presents the baseline model chose for the experiments, the procedure used to train the network and the parameters used. 

\subsection{The Baseline Model}

baseline model

\subsection{Experimental Procedure}

The procedure used to tune the performance of the network involved three stages. During the initial stage the network was trained for few epochs, with a coarse spread of hyper parameter and multiple algorithms were tested during this phase. The overall results from the initial trials informed the finer range of parameters used for additional training and further tuning of hyper parameters. The second stage involved longer training. The third stage, before the final testing, the model was trained for significantly longer, and only three varieties of the model were used. 

\subsubsection{Stage 1: Initial Training and Model Verification}

activation functions and initial learning rates

number of filter channels, and kernel sizes

batch normalization and l2 loss


\subsubsection{Stage 2: Fine-tuning hyper parameters} 

\subsubsection{Stage 2: Finding the one and Testing}
 
 
\section{Results and Discussion}

\section{Conclusion}

\section{Future Work}


\clearpage
\medskip
\bibliographystyle{IEEEtran}
\bibliography{ref.bib}

\end{document}
